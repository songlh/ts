%research.tex 26 October 2007
\documentclass[11pt]{proposal}
\usepackage{amsfonts}
\usepackage{amssymb}
%\usepackage{latexsym}
%\usepackage{graphicx}
\usepackage{fancyhdr}

\voffset=10mm%to print: dvips *.dvi; ps2pdf *.ps
\hoffset=-2mm
\oddsidemargin=18pt     \evensidemargin=18pt
\headheight=18pt        \topmargin=0pt
%\textheight=602--658pt \textwidth=436--437pt
\textheight=648pt       \textwidth=436pt
\parskip=2pt plus 1pt

\def\myname{\Large \bfseries E\normalsize ZRA \Large M\normalsize ILLER}

\pagestyle{fancy}
\cfoot{}				     % no footers (in pagestyle fancy) 
\lhead{{\noindent\myname}}		     % running left heading
\rhead{{\Large\bfseries \Large R\normalsize ESEARCH
	\Large S\normalsize TATEMENT
			\quad\Large\thepage}}% running right head

\newcommand{\comment}[1]{{$\star$\sf\textbf{#1}$\star$}}
\newcommand{\excise}[1]{}

\def\refname{{\Large \bf References}}
\newtheorem{thm}{Theorem}[section]
\newtheorem{conj}[thm]{Conjecture}
\newtheorem{prop}[thm]{Proposition}
\newtheorem{prob}[thm]{Problem}
\newtheorem{lemma}[thm]{Lemma}
\newtheorem{questions}[thm]{Questions}
\newtheorem{Defn}[thm]{Definition}

\newenvironment{defn}{\begin{Defn}\rm}{\end{Defn}}

% For proofs
\newenvironment{proof}{\begin{trivlist}\item {\it
	Proof.\,}}{\mbox{}~\hfill~$\Box$\end{trivlist}}

%For numbered lists with arabic 1. 2. 3. numbering
\newenvironment{numbered}%
        {\begin{list}
                {\noindent\makebox[0mm][r]{\arabic{enumi}.}}
                {\leftmargin=5.5ex \usecounter{enumi}
		 \topsep=1.5mm \itemsep=-1.4mm}
        }
        {\end{list}}

%For numbered lists within the main body of the text
\newenvironment{romanlist}%
        {\renewcommand\theenumi{\roman{enumi}}\begin{list}
                {\noindent\makebox[0mm][r]{(\roman{enumi})}}
                {\leftmargin=5.5ex \usecounter{enumi}
		 \topsep=1.5mm \itemsep=0mm}
        }
        {\end{list}\renewcommand\theenumi{\arabic{enumi}}}

%For sections within the bibliography
\def\bibsec#1{\item[#1]\vspace{.5ex}\item[]\vspace{-2.5ex}}

% bold sans-serif label for bibliography subsections
\def\bibsubsec#1{\item[\textsf{\hspace{-2.2ex}\textbf{#1}}]\vspace{.5ex}\item[]\vspace{-2.5ex}}

% marginal year labels for bibliography
\reversemarginpar
\def\bibyear#1{\makebox[-1ex][l]{}\marginpar[{\flushright{\small#1}}]{}}

%Single characters, used in math mode
\def\0{\mathbf 0}
\def\1{\mathbf 1}
\def\2{\mathbf 2}
\def\<{\langle}
\def\>{\rangle}
\def\CC{{\mathbb C}}
\def\EE{{\mathcal E}}
\def\FF{{\mathcal F}}
\def\GG{{\mathcal G}}
\def\HH{{\mathcal H}}
\def\MM{{\mathcal M}}
\def\NN{{\mathbb N}}
\def\QQ{{\mathbb Q}}
\def\RR{{\mathbb R}}
\def\ZZ{{\mathbb Z}}
\def\aa{{\mathbf a}}
\def\bb{{\mathbf b}}
\def\del{\partial}
\def\kk{\Bbbk}
\def\mm{{\mathfrak m}}
\def\nn{{\mathfrak n}}
\def\pp{{\mathfrak p}}
\def\qq{{\mathfrak q}}
\def\rr{{\mathbf r}}
\def\uu{{\mathbf u}}
\def\vv{{\mathbf v}}
\def\ww{{\mathbf w}}
\def\xx{{\mathbf x}}
\def\yy{{\mathbf y}}
\def\zz{{\mathbf z}}
\renewcommand\tt{{\mathbf{t}}}
% use \texttt{#1} to put #1 in typewriter font

\newcommand\KK{\hspace{.35ex}\ol{\hspace{-.35ex}K\hspace{-.05ex}}\hspace{.05ex}}
\newcommand\Kv{K_{\hspace{-.2ex}v}}
\newcommand\KKv{\KK_{\hspace{-.2ex}v}}

%roman font words for math mode and
%math symbols without arguments
\def\K{{$K$}}
\def\gl{{\mathit{G\!L}}}
\def\th{{\rm th}}
\def\dom{\backslash}
\def\fln{{{\mathcal F}\ell_n}}
\def\too{\longrightarrow}
\def\dash{\hspace{.1ex}{\raisebox{.2ex}{\underline{\ \,}}\hspace{.18ex}}}
\def\from{\leftarrow}
\def\into{\hookrightarrow}
\def\onto{\twoheadrightarrow}
\def\rank{{\mathrm{rank}}}
\def\spec{{\mathrm{Spec}}}
%\def\spot{{\hbox{\raisebox{1.7pt}{\large\bf .}}\hspace{-.5pt}}}
\def\spot{{\hbox{\raisebox{.33ex}{\large\bf .}}\hspace{-.5pt}}}
\def\minus{\smallsetminus}
\def\nothing{\varnothing}

%math symbols taking arguments
\def\IN#1{{\mathsf{in}_>#1}}
\def\ol#1{{\overline {#1}}}
\def\ub#1{\hbox{\underbar{$#1$\hspace{-.25ex}}\hspace{.25ex}}}
\def\wt#1{{\widetilde {#1}}}
\def\ubs#1{{\hbox{\underbar{$\scriptstyle#1$\hspace{-.25ex}}\hspace{.25ex}}}}


%%%%%%%%%%%%%%%%%%%%%%%%%%%%%%%%%%%%%%%%%%%%%%%%%%%%%%%%%%%%%%%%%%%%%%
\begin{document}%%%%%%%%%%%%%%%%%%%%%%%%%%%%%%%%%%%%%%%%%%%%%%%%%%%%%%
%%%%%%%%%%%%%%%%%%%%%%%%%%%%%%%%%%%%%%%%%%%%%%%%%%%%%%%%%%%%%%%%%%%%%%


%thispagestyle{plain}

\mbox{}
%smallskip
%vspace{-7ex}

\vspace{-4.3ex}
\centerline{\Large\bf Combinatorics, geometry, algebra, and applications}


%%%%%%%%%%%%%%%%%%%%%%%%%%%%%%%%%%%%%%%%%%%%%%%%%%%%%%%%%%%%%%%%%%%%%%
{}%%%%%%%%%%%%%%%%%%%%%%%%%%%%%%%%%%%%%%%%%%%%%%%%%%%%%%%%%%%%%%%%%%%%
%%%%%%%%%%%%%%%%%%%%%%%%%%%%%%%%%%%%%%%%%%%%%%%%%%%%%%%%%%%%%%%%%%%%%%


%%%%%%%%%%%%%%%%%%%%%%%%%%%%%%%%%%%%%%%%%%%%%%%%%%%%%%%%%%%%%%%%%%%%%%
%section{Introduction}%%%%%%%%%%%%%%%%%%%%%%%%%%%%%%%%%%%%%%%%%%%%%%%%
%vspace{-.5em}
\bigskip
\noindent
{\Large\bf Introduction}
\medskip


\label{s:intro}


\noindent
My research centers around combinatorial, computational, and
cohomological problems originating in geometry and algebra, with ties
to computer science.  Currently, my participation at the Institute for
Mathematics and its Applications (IMA) year on Mathematics of
Molecular and Cellular Biology is stimulating potential interactions
with geometric methods in biology and statistics (not to be confused
with algebraic statistics and the related questions in biology, which
were topics of last year's IMA program).

The unifying idea in my research has been to isolate or exploit
combinatorial structures that govern or arise from continuous
contexts.  For example, if a continuous process carries underlying
discrete data, then those data might be harnessed to produce
algorithms for the continuous process.  On the other hand, the goal
could also be to understand the combinatorics rather than the
geometry; the geometry then serves as a vehicle for interpolating
between different interpretations of the combinatorics.

The aim of this exposition is to give a broad
perspective on more specific areas in which I have worked and am
currently interested.  It is broken into four sections:
% \noindent
% \setcounter{tocdepth}{1}
% \tableofcontents
\begin{list}{\arabic{enumi}.}
           {\leftmargin=5ex \rightmargin=2ex \usecounter{enumi}
	    \itemsep=-1.5mm \topsep=-1.5mm}
\item
{\bf Metric polyhedral geometry}
	\hfill\pageref{s:metric}
\item
{\bf Combinatorial positivity in algebraic geometry}
	\hfill\pageref{s:positivity}
\item
{\bf Commutative and homological algebra}
	\hfill\pageref{s:commalg}
\item
{\bf Interdisciplinary projects}
	\hfill\pageref{s:inter}
\item
{\bf References}
	\hfill\pageref{s:refs}
\end{list} \vspace{1.5mm}
These sections do not cover my expository works.  These include two
graduate textbooks written: one recently completed with a number of
coauthors on the muti-faceted topic of local
cohomology~\cite{24hours}, and another with Bernd Sturmfels
summarizing a decade's worth of developments in combinatorial
commutative algebra \cite{cca}.  With Sturmfels and Vic Reiner, I
edited the volume \cite{pcmi2004} of graduate courses from the PCMI
Summer School on Geometric Combinatorics, which the three of us also
organized; Vic Reiner and I wrote the overview~\cite{overview}.  In
connection with the MSRI year on Commutative Algebra, I wrote an
article on Hilbert schemes of points in the plane \cite{hilbAppendix},
as part of my duties as a TA for Mark Haiman's short course.
Subsequent to a COCOA (Computation in Commutative Algebra) conference,
I wrote a long article on monomial ideals with David Perkinson
\cite{eightLect}.  Finally, in connection with a lecture series I gave
at CRM Montr\'eal, I am in the process of writing an article with
Laura Matusevich as well as two of the students there, Huilan Li and
Craig Sloss, on connections between binomial ideals and hypergeometric
systems.


%end{section}%%%%%%%%%%%%%%%%%%%%%%%%%%%%%%%%%%%%%%%%%%%%%%%%%%%%%%%%%
%%%%%%%%%%%%%%%%%%%%%%%%%%%%%%%%%%%%%%%%%%%%%%%%%%%%%%%%%%%%%%%%%%%%%%
%%%%%%%%%%%%%%%%%%%%%%%%%%%%%%%%%%%%%%%%%%%%%%%%%%%%%%%%%%%%%%%%%%%%%%
\section{Metric polyhedral geometry}%%%%%%%%%%%%%%%%%%%%%%%%%%%%%%%%%%


\label{s:metric}


This project is currently undergoing rapid development, with two
papers in progress \cite{unfolding,complexity}.  The stage for this
research is set by joint work of mine with Igor Pak \cite{fold}.
Motivated by relations to a number of classical algorithmic problems
in discrete and computational geometry, our fundamental observation
there is that convexity and polyhedrality together impose rich
combinatorial structures on the collection of shortest paths in a
metrized sphere.  The methods of metric combinatorics constitute a
blend of differential geometry, classical polyhedral geometry, poset
combinatorics, geometric topology, and complexity theory.

The first main result of \cite{fold} says that the boundary~$S$ of any
$(d+1)$-polytope has a certain polyhedral nonoverlapping unfolding
into~$\RR^d$.  More precisely, the exponential map $T_v \to S$ from
the tangent space at any generic point $v \in S$ is surjective, and
the set of tangent vectors exponentiating to shortest paths
(length-minimizing geodesics) is a closed, star-shaped, polyhedral set
in~$T_v$, called the \emph{source foldout}.  Equivalently, the cut
locus $\KKv \subseteq S$ for the source point~$v$ is a polyhedral
complex of pure dimension $d-1$ (though not a subcomplex of~$S$), and
slicing~$S$ open along~$\KKv$ results in a polyhedral ball that can be
laid flat in~$\RR^d$ without overlap.  This construction had been
known in dimension $d=2$ \cite{VP71,SS}, but there seems to have been
no known unfolding at all for $d \geq 3$.

The second main result of \cite{fold} is an effective algorithm for
constructing the source foldout.  This involves imposing---and
subsequently navigating---certain order-theoretic combinatorial
structures on the set of shortest paths, using iterated tangent data
from an expanding wavefront.  Currently, I am advising an undergradute
(UROP) student, Nate Born, in a project to implement our algorithm;
see Section~\ref{cs} for the importance of doing this.  The main
mathematical open question from \cite{fold} is the complexity of the
algorithm: although Pak and I were able to prove that it is polynomial
in the size of the output, we could only conjecture that the output is
polynomial in the number of facets of~$S$.  (It is exponential in the
dimension~$d$, of course.)  I wish to stress that this is a natural,
purely geometric question, and does not depend on the nature of our
algorithm: given a point~$v$ on a polyhedral manifold~$S$ and a
facet~$F$ of~$S$, how many shortest paths start at~$v$ and end in~$F$?
The answer can be exponential in the number of facets if $S$ is
allowed to have negative curvature, even if $S$ is a polyhedral
$2$-sphere, but nonetheless the number of shortest paths from~$v$
to~$F$ should be polynomial in the number of facets for convex
polyhedral spheres.

Before describing my approach to the metric complexity of polyhedral
spheres, it will help to review one more observation from~\cite{fold}.
In dimension $d=2$, there is a second \emph{Alexandrov unfolding}
\cite{Ale48} of a polyhedral sphere~$S$, in a sense dual to the source
foldout: fix any source point $v \in S$, and slice $S$ open along the
shortest paths from~$v$ to the vertices of~$S$.  That the resulting
locally flat disk can be globally laid flat in~$\RR^2$ without overlap
was proved by Aronov and O'Rourke \cite{AO92}.  However, the
construction itself fails in dimensions $d \geq 3$: the union of the
shortest paths from a fixed source point~$v$ to the $(d-2)$-skeleton
is not contractible.  Roughly speaking, the angles of the shortest
paths to (say) an edge~$E$ make a quantum leap as they swing past any
$(d-2)$-face between~$v$~and~$E$.

The first part \cite{unfolding} of my current work aims to demonstrate
that, nonetheless, Alexandrov unfoldings generalize naturally to
dimensions $d \geq 3$.  The idea is to consider the \emph{gradient
vector field} on~$S$, which points in the direction of steepest
distance increase from~$v$.  That there even exists a unique such
vector at each point of~$S$ is already nontrivial, and fails
immediately for any polyhedral manifold exhibiting negative curvature
of any sort.  More surprising still is the observation that this
distinctly discontinuous vector field should determine a continuous
flow on~$S$ away from~$v$.  For a generic source point~$v$, the
\emph{Alexandrov slice set} is the union of the flow lines into the
$(d-2)$-skeleton of~$S$; in dimension $d=2$, this is precisely the
Alexandrov slice set from before, but in higher dimensions, it fills
in the gaps left after joining $v$ to the $(d-2)$-skeleton by shortest
paths.  In general, this slice set is contractible because the
(continuous!)\ gradient flow pushes its complement onto a retract of
the cut locus~$\KKv$.  Which retract?  The one obtained by gradient
flow on~$\KKv$~itself.

Aside from generalizing a classical $d=2$ construction to arbitrary
dimension, why are Alexandrov unfoldings by gradient flow important?
Simply put, the combinatorics of the Alexandrov slice set controls the
metric complexity of the cut locus.  Making this statement precise is
the goal of the second half \cite{complexity} of my current work.  The
intuition is as follows.  Imagine that $S$ is your universe, and you
are standing at the source point~$v$.  As you look out, your horizon
is studded with pieces of $(d-2)$-faces, where there is nontrivial
curvature.  These pieces don't subdivide your horizon into regions
(this is why the union of shortest paths isn't a slice set); however,
the union of the gradient flow lines into the boundary of each facet
should.

In dimension $d=3$, the picture is just small enough to be describable
in everyday terms: your horizon is a $2$-sphere, and the flow lines to
the boundary of a single facet of~$S$ should pierce the sphere in a
polygon.  Taking all of the facets at once, you see a constellation of
polygons.  These subdivide your horizon into regions, and we need the
number of these regions to be polynomial in the number of facets.  The
underlying structure, which I am developing for~\cite{complexity}, is
a certain generalization of oriented matroids, the motivation being
that the number of regions in a hyperplane arrangement is polynomial
in the number of hyperplanes (but exponential in the dimension~$d$, of
course).

An oriented matroid can be represented as a collection of
pseudospheres, each of which divides the horizon into two pieces.
Here, instead, we have a collection of \emph{hedroids}, the
flow-projections of the edges (in $d=3$; for higher~$d$, these will be
faces of dimension $d-2$), each of which only locally divides the
horizon into two pieces.  The \emph{polyhedroid} structure, which
generalizes the notion of oriented matroid, amounts to control over
how these hedroids are allowed to meet.  In a matroid, the nonempty
intersections of pseudospheres look locally like oriented matroids of
smaller dimension; in a polyhedroid, these intersections are required
to look like arbitrary polyhedral fans (or perhaps like general
polyhedroids) of smaller dimension.  The goal of~\cite{complexity},
then, is twofold: first, develop a theory of polyhedroids enough to
conclude polynomiality for the number of regions; and second, prove
that the Alexandrov slice set is a polyhedroid.  This latter part, if
true, would encode fundamental geometric statements about the
collections of shortest paths in a convex polyhedral sphere.


Where, in general, do I see the program of flows on polyhedral spaces
going?  In geometric topology, \emph{entropy} is a measure of the
exponential geodesic complexity of a manifold \cite{manning79}.  In
that context, however, entropy is always zero for positively curved
manifolds.  Here, in contrast, the facial structure provides us with
growth rate measures finer than the topological ones,
% after all, boundaries of polytopes are topologically pretty
% simple.  *careful: homotopy groups of spheres are pretty
% interesting; the point here is that geodesic complexity is a
% \pi_1 thing, not a general homotopy group thing*
allowing us to make meaningful statements about subexponential numbers
of combinatorial types of geodesics.  I wonder what can be said about
metric complexity for geodesics that aren't shortest paths, or on
general polyhedral manifolds; for example, is there anything between
polynomial and exponential growth?  Still in the topological realm,
a~number of geometrically defined flows in polyhedral geometry could
potentially be viewed as flows on higher-dimensional polyhedral
spaces; for example, consider the combinatorial Yamabe \cite{glick05},
Ricci \cite{CL03}, or scalar curvature flows on polyhedral
$3$-manifolds (the latter having been used by Bobenko and Izmestiev
\cite{BI06} to give an effective algorithmic version of Alexandrov's
embedding theorem for convex polyhedral $2$-spheres).  In another
direction, it was noticed already by Volkov \cite{Vol68} (or see
\cite[Section~12.1]{Ale50} for a translation) that the intrinsic
metric on a convex polyhedral sphere~$S$ can be intimately tied to the
extrinsic geometry of its convex hull in dimension $d=2$.  It would be
very interesting to see what can be said in higher dimensions.  In any
case, I find our limited understanding of the metric geometry of
convex polyhedra in general to be an impediment to progress on the
most basic of questions, even in $d=2$: do polyhedra admit
nonoverlapping foldouts by slicing along ridges---that is, faces of
dimension $d-1$?


%end{section}%%%%%%%%%%%%%%%%%%%%%%%%%%%%%%%%%%%%%%%%%%%%%%%%%%%%%%%%%
%%%%%%%%%%%%%%%%%%%%%%%%%%%%%%%%%%%%%%%%%%%%%%%%%%%%%%%%%%%%%%%%%%%%%%
%%%%%%%%%%%%%%%%%%%%%%%%%%%%%%%%%%%%%%%%%%%%%%%%%%%%%%%%%%%%%%%%%%%%%%
\section{Combinatorial positivity in algebraic geometry}%%%%%%%%%%%%%%


\label{s:positivity}


One of the major ways in which combinatorics interacts with algebraic
geometry is through questions of positivity.  In my
work~\cite{grobGeom} with Knutson, we provide a proof by geometric
degeneration that the coefficients of Schubert polynomials are
positive.  These polynomials represent the classes Poincar\'e dual to
Schubert varieties in the cohomology ring of the flag variety.  Their
coefficients had been known since their inception \cite{LSpolySchub}
to be positive, but the connection of these particular polynomial
representatives to geometry had been more tenuous.  In contrast, the
polynomials we prove geometrically to be positive in joint work
additionally with Shimozono~\cite{quivers} were not known to be
so---by combinatorics or by any other means.  In particular, one of
our four formulae proves the essence of the main Conjecture of Buch
and Fulton in~\cite{BF99}.

The first part of my work~\cite{flagTab} with Knutson and Yong brings
the combinatorics of decompositions of simplicial complexes to bear on
the algebraic geometry of degenerations.  In particular, we introduce
\emph{geometric vertex decomposition}, a procedure to squeeze
algebraic varieties onto coordinate subspaces one by one.  The rest is
an embellishment on~\cite{grobGeom}: we use our new technology to show
that degenerating by an opposite torus weight---equivalently, using a
diagonal term order instead of an antidiagonal one---yields different
combinatorial formulae for Schubert polynomials when the permutation
is grassmannian.  In another offshoot of~\cite{grobGeom}, Kogan and I
show that under the Gonciulea--Lakshmibai degeneration~\cite{GL96} of
the flag variety to a toric variety, the Schubert varieties degenerate
to reduced unions of toric faces~\cite{deform} combinatorially
identifiable as summands in Schubert polynomials.

In $K$-theory, as opposed to cohomology, positivity really means
sign-alternation: for some fixed degree~$c$, all of the coefficients
in degree $c+i$ have sign $(-1)^i$.  Knutson and I observed in 2000,
while working on~\cite{grobGeom}, that this phenomenon for $K$-classes
of Schubert varieties arises from Cohen--Macaulay initial ideals by
Alexander duality.  A different sign alternation was conjectured by
Buch \cite{buchLR} for multiplication in the ordinary $K$-theory of
flag varieties with its basis of Schubert structure sheaves, and it
was proved by Brion \cite{brion02}.  Brion's proof relied on a certain
transversality theorem for $K$-theory on homogeneous spaces.  A~number
of other results in Schubert calculus depend on related (but not
equal) versions of this transversality.  To avoid endlessly proving
minor extensions of the known such Kleiman--Bertini type results,
Speyer and I proved a rather general version of vanishing for sheaf
Tor on homogeneous spaces~\cite{torVanish}, which holds even in
positive characteristic.  As general as Brion's sign alternation
theorem is, it does not cover the alternation in Buch's conjecture
\cite{buch02} for the expansion of quiver $K$-polynomials in terms of
Grothen\-dieck polynomials for partitions.  This required separate
proofs, which Buch \cite{buchAlt} and~I~\cite{kquiv}
\mbox{independently produced}.

In a current project with Stephen Griffeth, a postdoc I am mentoring
at Minnesota (and former student at Wisconsin under Arun Ram), we are
thinking about sign alternation in \emph{equivariant} $K$-theory of
homogeneous spaces.  We intend to apply ideas from equivariant Chow
theory combined with Brion's sign alternation theorem in ordinary
$K$-theory.  If this works, it will generalize Graham's positivity
\cite{grahamPos} in the case of equivariant cohomology for flag
varieties.

Sometimes algebraic geometry like that above can give rise to purely
combinatorial problems---or solutions.  In the course of our study of
Schubert polynomials, Knuston and I found a positive combinatorial
rule for generating Schubert polynomials, and I subsequently used it
to give a quick positive construction of Schubert polynomials from
scratch~\cite{mitosis}.  A~student, Ning Jia, whom I mentored at
Minnesota, gave an elegant simplification \cite{njia} of our
proof~\cite{grobGeom} of duality between certain combinatorial objects
associated to Schubert polynomials (my name is on \cite{njia} for a
small contribution only because she insisted).  Following up on our
proof of the Cohen--Macaulay condition for certain simplicial
complexes in~\cite{grobGeom}, Knutson and I discovered a general
topological way to view the exchange axiom in an arbitrary Coxeter
group~\cite{subword}: given a fixed word, the set of subwords with a
given product forms a ball or sphere.  These \emph{subword complexes}
include, as special cases, certain simplicial complexes whose facets
are the semistandard Young tableaux of a given shape and whose
interior faces are the semistandard set-valued tableaux \cite{buchLR}.
This suggested a different, purely combinatorial (poset-theoretic)
construction of a family of nonisomorphic simplicial complexes with
the same properties \cite{tabComplex}.


%end{section}%%%%%%%%%%%%%%%%%%%%%%%%%%%%%%%%%%%%%%%%%%%%%%%%%%%%%%%%%
%%%%%%%%%%%%%%%%%%%%%%%%%%%%%%%%%%%%%%%%%%%%%%%%%%%%%%%%%%%%%%%%%%%%%%
%%%%%%%%%%%%%%%%%%%%%%%%%%%%%%%%%%%%%%%%%%%%%%%%%%%%%%%%%%%%%%%%%%%%%%
\section{Commutative and homological algebra}%%%%%%%%%%%%%%%%%%%%%%%%%


\label{s:commalg}


Most of my ealiest work, starting in graduate school, concerned
combinatorial methods in commutative and homological algebra.  Many
themes along these lines persist in my research to this day.  To make
the descriptions easier, I have (more or less arbitrarily) split the
exposition into three subsections.


\subsection{Multivariate hypergeometric systems}%%%%%%%%%%%%%%%%%%%%%%


Hypergeometric mathematics, of the type arising in my research,
combines algebraic geometry, lattice point combinatorics, binomial
commutative algebra, homological and noncommutative algebra of
differential operators, and complex analysis.

Hypergeometric series are power series whose adjacent coefficients are
related by rational functions.  In 1889, J.~Horn introduced bivariate
versions of the classical univariate hypergeometric
series~\cite{horn89}.  In the 1940s, Erd\'elyi discovered sporadic
solutions to Horn's hypergeometric systems~\cite{erdelyi}.  In work
with Alicia Dickenstein and Laura Matusevich~\cite{horn}, we explain
these extra solutions as consequences of binomial primary
decomposition, which we describe combinatorially in terms of lattice
points.  The specific form of our primary decompositions, which refine
those of Eisenbud and Sturmfels \cite{binomialIdeals}, allow us to
reduce---by filtration methods---to the toric hypergeometric
techniques developed by Gelfand, Graev, Kapranov, Zelevinsky, and
others, particularly Adolphson, in the 1980's and 1990's
\cite{GGZ87,GKZ89,Ado94}.

Our work on Horn systems relied in an absolutely essential way on the
homological \emph{Euler--Koszul}\/ technology introduced with
Matusevich and Walther \cite{rankJumps}.  The main purpose
of~\cite{rankJumps} was to prove a conjecture by Sturmfels on the
equivalence of the Cohen--Macaulay condition with the lack of
rank-jumping parameters.  We accomplished this using sheaf-theoretic
techniques on the space of parameters.  We identified the rank-jump
locus as an algebraic variety, defined via sheaf cohomology of the
associated toric variety, whose non-emptiness witnesses the failure of
the Cohen--Macaulay condition.  In the case where the toric variety is
simplicial, Matusevich and I had proved this connection between local
cohomology and rank jumps by a direct polyhedral
argument~\cite{simpJumps}.

Currently, I am guiding one of my graduate students, Robert Edman, in
his chosen project to understand Kapranov's analogues of
hypergeometric systems for reductive groups (instead of tori).  This
research involves the same list as ordinary hypergeometric
mathematics, except that lattice point combinatorics (a.k.a.\
representation theory of a torus) is replaced with deeper topics in
representation theory for reductive groups.


\subsection{Homological combinatorics: free and injective resolutions}


Much of the work in my dissertation \cite{emThesis} concerned
combinatorial aspects of homological algebra in situations that are
\emph{finely graded}: the ambient ring is graded in such a way that
its components have vector space dimension~$1$.  Over polynomial
rings, I made Alexander duality, an a~priori combinatorial topological
operation, into a functor~\cite{alexDual} that interchanges free and
injective resolutions.  I realized later~\cite{gm} that my functorial
Alexander duality for resolutions, when considered in the derived
category, is a special case of finely graded Greenlees--May
duality~\cite{gm92}.

Duality for resolutions allowed me to homologically interpret the
Eagon--Reiner duality \cite{ER98} between the Cohen--Macaulay and
linear resolution conditions for squarefree monomial ideals, and
thereby to generalize it to affine semigroup rings~\cite{cmQuotients}.
The appearance of injective resolutions---and their finely graded
truncations, the \emph{irreducible resolutions}, which I
introduced---indicated their utility in computations; hence David Helm
and I produced an algorithm to calculate them~\cite{injAlg}.  Part of
this algorithm relies on a bound from~\cite{bassNumbers}, obtained in
an abstract homological setting, on how the finely graded shifts of
the indecomposable summands in an injective resolution vary from prime
to prime.  The purpose of~\cite{bassNumbers}, however, was to prove
the finiteness of Bass numbers for local cohomology over simplicial
semigroup rings, and the failure in the nonsimplicial case.  This
provided a general toric framework for Hartshorne's counterexample
\cite{HarCofinite} to Grothendieck's conjecture on this finiteness
issue.

My most recent research on resolutions is joint work
\cite{multIdealSums} with a bright graduate student, Shin-Yao Jow, of
Rob Lazarsfeld's at Michigan.  Shin-Yao's idea was to provide a
geometric, sheaf-theoretic (as opposed to characteristic~$p$
algebraic) proof of a formula for the multiplier ideals of a sum of
ideal sheaves.  We reduced the problem to an analytically local one on
a resolution of singularities, where it became a mixture of
combinatorial commutative algebra, particularly cellular resolutions
of monomial ideals, and simplicial topology, particularly
homology-manifolds-with-boundary.


\subsection{Monomial ideals and related combinatorics}%%%%%%%%%%%%%%%%


Special classes of monomial ideals give rise to particularly beautiful
and useful combinatorics.  This is the case, for example, with
\emph{generic} monomial ideals (\cite{BPS98} and \cite{generic}),
whose exponent vectors are, in effect, random.  These ideals are
characterized in numerous ways~\cite{generic}, particularly by the
invariance of their minimal free and injective resolutions (which are
explicit convex-geometric simplicial complexes) under deformations of
the exponents of the generators.  Similarly, three-dimensional
monomial ideals give rise to pretty staircase pictures, in which the
minimal resolutions can be drawn explicitly as planar
graphs~\cite{planarGraphs,AAECC}.

Generalizations of monomial ideals can also carry simplicial or
polyhedral combinatorics.  This well-known phenomenon is illustrated
by my joint work with Reiner~\cite{simplicialPoset,reciprocal}.  The
first of those concerns ideals of simplicial posets, which are certain
flexible weakenings of simplicial complexes; we simplify Masuda's
proof \cite{masuda03} of Stanley's conjecture \cite{Sta91} concerning
the $h$-vectors of these objects.  The second concerns certain radical
monomial ideals in semigroup rings; we generalize and place in a
(local co)homological framework one of Stanley's reciprocity theorems
\cite{Sta74}.

% \cite{multiplicities}

I have recently begun working with Raman Sanyal, a graduate student
under G\"unter Ziegler at Technische Universit\"at Berlin, on
``combinatorial moduli spaces'' for resolutions of monomial ideals.
The basic idea is to classify commutative polynomial ideals generated
by $n$ monomials in $d$ variables according to their minimal free or
injective resolutions.  In slightly more detail, such an ideal is
determined by choosing a $d \times n$ matrix of nonnegative integers,
the columns being the exponent vectors of the monomial generators.
However, the homological properties of the ideal depend only on
order-theoretic combinatorial data extracted from the integer matrix,
such as the permutation of the columns required to make a given row
weakly increasing.  The goal is to describe the subdivision of the
space of matrices into the equivalence classes consisting of the
monomial ideals with identical homological properties.  For example,
the generic monomial ideals (\cite{BPS98} and \cite{generic})
constitute the maximal regions in the subdivision.  The eventual goal
would be to attack the (apparently much harder) analogue for lattice
ideals; see \cite{PS98}, where the generic ideals are identified.  For
the time being, we are starting with the case $d = 3$, where planar
graphs help immensely~\cite{planarGraphs}.


%end{section}%%%%%%%%%%%%%%%%%%%%%%%%%%%%%%%%%%%%%%%%%%%%%%%%%%%%%%%%%
%%%%%%%%%%%%%%%%%%%%%%%%%%%%%%%%%%%%%%%%%%%%%%%%%%%%%%%%%%%%%%%%%%%%%%
%%%%%%%%%%%%%%%%%%%%%%%%%%%%%%%%%%%%%%%%%%%%%%%%%%%%%%%%%%%%%%%%%%%%%%
\section{Interdisciplinary projects}%%%%%%%%%%%%%%%%%%%%%%%%%%%%%%%%%%


\label{s:inter}


A good deal of my previous work has overt connections to computer
science; see Section~\ref{cs}.  More recently, a triad of independent
and nearly simultaneous circumstances have placed me in an ideal
position to foster immediate collaborations with scientists in
disparate areas of mathematics as well as theoretical and more
traditional experimental sciences.  The triad of events consists of
(i)~Minnesota's recent hire of Gilad Lerman, an applied mathematician,
with whom I have been having tantalizing conversations; (ii)~my
introduction to R.~Dennis Cook, an established statistician at
Minnesota specializing in diagnostics (finding anomalies in data); and
(iii)~the current year-long program on Mathematics of Molecular and
Cellular Biology at the Institute for Mathematics and its Applications
(IMA).  It is becoming increasingly clear to me that my work in
geometry---polyhedral as well as algebraic---has great potential for
these collaborations.

The projects in this section are at varying developmental stages;
because of the timing, many have not yet coalesced into concrete lines
of investigation.

\subsection{Computer science}\label{cs}%%%%%%%%%%%%%%%%%%%%%%%%%%%%%%%

A number of times in the past, I have found myself working in areas
directly influeced by explicit algorithms for various kinds of
computations: in algebra (of resolutions~\cite{injAlgMEGA,AAECC} and
of hypergeometric series \cite{hornMEGA}), in combinatorics (of planar
graphs \cite{planarGraphs,AAECC}; see also the works of Felsner
\cite{felsner01,felsner03}), and in geometry (of
polyhedra~\cite{fold}).  In particular, my
work~\cite{injAlg,injAlgMEGA} with Helm was the first algorithm for
computing local cohomology over any class of singular rings, and my
work with Pak was the first algorithm to unfold convex polyhedra in
any dimension $d \geq 3$.  It is the latter where I see important
advances in the near future.  Looking back at Section~\ref{s:metric},
once the Alexandrov unfolding is defined for dimensions $d \geq 3$, it
will be important to have an algorithm for producing it.  Indeed, it
is natural to conjecture that the unfolding will be nonoverlapping, as
it is for $d=2$ \cite{AO92}, and computers are the best way to produce
copious evidence (or a counterexample).  This brings me back to the
undergraduate research project I am conducting with Nate Born: if I
understand things correctly, then it should be quite easy to rearrange
pieces of the source foldout, which Nate is computing, to get the
Alexandrov unfolding.  This will truly be a computational geometry
experiment, and I am hoping to get people like O'Rourke and Joe
Mitchell~\cite{MMP87} interested in it.

\subsection{Statistics}\label{stat}%%%%%%%%%%%%%%%%%%%%%%%%%%%%%%%%%%%

At a meeting to discuss future programs for the IMA, I began
conversations with R.~Dennis Cook, a statistician here at Minnesota,
on fitting data with nonlinear geometric structures.  These structure
could be real algebraic varieties, or polyhedral complexes, or
something else---we'll have to see what's possible abstractly, what's
desirable, and what's computationally feasible.  (But first, we'll
have to teach each other our respective subjects!)  In contrast with
the field that has come to be known as ``algebraic statistics'', in
which algebra or algebraic geometry describes
% for the parameter spaces that are
statistical models (allowed sets of probabilities for a fixed set of
random variables), Cook and I are talking about the completely
different issue of methods to make geometric sense of data sets.

\subsection{Geometric measure theory}\label{gilad}%%%%%%%%%%%%%%%%%%%%

In conversations with Gilad Lerman and his student,
J.\thinspace{}Tyler Whitehouse, we have reduced certain questions in
geometric measure theory to various kinds of high-dimensional metric
polyhedral geometry of the sort that I have been thinking about
recently.  A~typical problem in this area is how to place a hyperplane
so as to make it pass within $\varepsilon$ of a given set of points.
Gilad and I have plans to pursue further conversations to mine each
other's expertise.

\subsection{Mathematical biology}\label{bio}%%%%%%%%%%%%%%%%%%%%%%%%%%

During high-school, I lived in the vicinity of Bethesda, MD, and I was
lucky enough to spend four summers working at the National Institutes
of Health [NIH] there, working in laboratories on questions in
biochemistry and molecular biology.  My career goals turned later to
mathematics, but I have retained and nurtured my interest in the
biological sciences.  For some time, now, I have been looking for an
opportunity to integrate these interests of mine, and this year's
program at the IMA is it.  It is still early in the year, but already
I am seeing connections to my work on metric geometry.  For example,
flows on polyhedral spaces and differential inclusions are useful in
modeling self-regulating network models;
% involving chemical concentrations, sliding domains, ... etc.
see \cite{CdG06} for an instance of this.


%end{section}%%%%%%%%%%%%%%%%%%%%%%%%%%%%%%%%%%%%%%%%%%%%%%%%%%%%%%%%%
%%%%%%%%%%%%%%%%%%%%%%%%%%%%%%%%%%%%%%%%%%%%%%%%%%%%%%%%%%%%%%%%%%%%%%
\begin{excise}{
%%%%%%%%%%%%%%%%%%%%%%%%%%%%%%%%%%%%%%%%%%%%%%%%%%%%%%%%%%%%%%%%%%%%%%
\section{...}%%%%%%%%%%%%%%%%%%%%%%%%%%%%%%%%%%%%%%%%%%%%%%%%%%%%%%%%%


\label{s:...}

...

\subsection{...} \label{sub:}
%
...

\subsection{...} \label{sub':}
%
...

\paragraph{...}
...

\paragraph{...}
...


%end{section}%%%%%%%%%%%%%%%%%%%%%%%%%%%%%%%%%%%%%%%%%%%%%%%%%%%%%%%%%
%%%%%%%%%%%%%%%%%%%%%%%%%%%%%%%%%%%%%%%%%%%%%%%%%%%%%%%%%%%%%%%%%%%%%%
}\end{excise}
%%%%%%%%%%%%%%%%%%%%%%%%%%%%%%%%%%%%%%%%%%%%%%%%%%%%%%%%%%%%%%%%%%%%%%
%\footnotesize
\small
%\pagebreak
\section*{References}\label{s:refs}
\vspace{1.25ex}
%\bibliographystyle{amsalpha}\bibliography{biblio}\begin{excise}{
\begin{thebibliography}{33}


%%%%%%%%%%%%%%%%%%%%%%%%%%%%%%%%%%%%%%%%%%%%%%%%%%%%%%%%%%%%%%%%%%%%%%
\bibsec{\textbf{Publications and preprints by Ezra Miller}
% (rougly in reverse chronological order)
(numbered as in~CV)}

%%%%%%%%%%%%%%%%%%%%%%%%%%%%%%%%%%%%%%%%%%%%%%%%%%%%%%%%%%%%%%%%%%%%%%
\bibsubsec{\sf Books and expository articles}%%%%%%%%%%%%%%%%%%%%%%%%%
%%%%%%%%%%%%%%%%%%%%%%%%%%%%%%%%%%%%%%%%%%%%%%%%%%%%%%%%%%%%%%%%%%%%%%

\bibitem{24hours}\bibyear{2007}
(book with Srikanth Iyengar, Graham Leuschke, Anton Leykin, Claudia
	Miller, Anurag Singh, and Uli Walther) \emph{Twenty-four hours
	of local cohomology}, Graduate Studies in Mathematics,
	American Mathematical Society, Providence, RI.
	(xvi+282~pages)

\bibitem{pcmi2004}
(book edited with Vic Reiner and Bernd Sturmfels) \emph{Geometric
	Combinatorics}.  Lecture notes from the Graduate Summer
	School of the Institute for Advanced Study/Park City
	Mathematics Institute held in Park City, UT, July 11--31,
	2004.  IAS/Park City Math.\ Series, American Math.\ Society,
	Providence, RI; Institute for Advanced Study (IAS), Princeton,
	NJ.  (xvi+691~pages)

\bibitem{overview}
(with Vic Reiner) \emph{What is geometric combinatorics?}  In Geometric
	combinatorics (Park City, UT, 2004), IAS/Park City Math.\
	Series, American Math.\ Society, Providence, RI, pp.~1--17.

\bibitem{cca}\bibyear{2004}
(book with Bernd Sturmfels) \emph{Combinatorial Commutative Algebra},
	Graduate Texts in Mathematics Vol. 227, Springer--Verlag, New
	York.  (xiv+417~pages)

\bibitem{hilbAppendix}
\emph{Hilbert schemes of points in the plane}, Appendix to
	\emph{Commutative algebra of $N$ points in the plane}, by
	Mark Haiman, in Luchezar Avramov
%	, M.~Green, C.~Huneke, K.~Smith, and B.~Sturmfels
	et al., (eds.), \emph{Trends in Commutative Algebra}, MSRI
	Publications Vol.~51, Cambridge University Press, New York,
	pp.~153--180.

\bibitem{eightLect}\bibyear{2001}
(with David Perkinson) \emph{Eight lectures on monomial ideals}, in
	Queen's Papers in Pure and Applied Mathematics, no.~120,
	3--105.

%%%%%%%%%%%%%%%%%%%%%%%%%%%%%%%%%%%%%%%%%%%%%%%%%%%%%%%%%%%%%%%%%%%%%%
\bibsubsec{\sf Articles in progress}%%%%%%%%%%%%%%%%%%%%%%%%%%%%%%%%%%
%%%%%%%%%%%%%%%%%%%%%%%%%%%%%%%%%%%%%%%%%%%%%%%%%%%%%%%%%%%%%%%%%%%%%%

\bibitem{unfolding}
\emph{Metric combinatorics of convex polyhedra, II: gradient flow and
	Alexandrov unfolding}.

\bibitem{complexity}
\emph{Metric complexity of convex polyhedra}.

\bibitem{binomialCRM}
(with Huilan Li, Laura Matusevich, and Craig Sloss) \emph{Multivariate
	hypergeometric functions and binomial ideals}. 

%%%%%%%%%%%%%%%%%%%%%%%%%%%%%%%%%%%%%%%%%%%%%%%%%%%%%%%%%%%%%%%%%%%%%%
\bibsubsec{\sf Submitted journal articles}%%%%%%%%%%%%%%%%%%%%%%%%%%%%
%%%%%%%%%%%%%%%%%%%%%%%%%%%%%%%%%%%%%%%%%%%%%%%%%%%%%%%%%%%%%%%%%%%%%%

\bibitem{horn}
(with Alicia Dickenstein and Laura Matusevich) \emph{Binomial
	$D$-modules}, 47 pages.  \textsf{arXiv:math.AG/ 0610353}

\bibitem{njia}
(with Ning Jia) \emph{Duality of antidiagonals and pipe dreams},
	5~pages.  \textsf{arXiv:math.CO/0706.3031}

%%%%%%%%%%%%%%%%%%%%%%%%%%%%%%%%%%%%%%%%%%%%%%%%%%%%%%%%%%%%%%%%%%%%%%
\bibsubsec{\sf Peer-reviewed journal articles}%%%%%%%%%%%%%%%%%%%%%%%%
%%%%%%%%%%%%%%%%%%%%%%%%%%%%%%%%%%%%%%%%%%%%%%%%%%%%%%%%%%%%%%%%%%%%%%

\bibitem{multIdealSums}\bibyear{to appear}
(with Shin-Yao Jow) \emph{Multiplier ideals of sums via cellular
	resolutions}, Mathematical Research Letters, 15~pages.
	\textsf{arXiv:math.AG/ 0703299}

\bibitem{flagTab}
(with Allen Knutson and Alex Yong) \emph{Gr\"obner geometry of vertex
	decompositions and of flagged tableaux}, Journal f\"ur die
	reine und angewandte Mathematik, 23 pages.
	\textsf{arXiv:math.CO/0502144}

\bibitem{torVanish}
(with David Speyer) \emph{A Kleiman--Bertini theorem for sheaf tensor
	products}, Journal of Algebraic Geometry, 5~pages.
	\textsf{arXiv:math.AG/0601202}

\bibitem{tabComplex}
(with Allen Knutson and Alex Yong) \emph{Tableau complexes}, Israel
	Journal of Math., 18~pages. \textsf{arXiv:math.CO/0510487}

\bibitem{fold}\bibyear{2006}
(with Igor Pak) \emph{Metric combinatorics of convex polyhedra: cut
	loci and nonoverlapping unfoldings}, Discrete and
	Computational Geometry (electronic), DOI:
	10.1007/s00454-006-1249-0, pages OF1-OF50.
	\textsf{arXiv:math.MG/0312253}

\bibitem{quivers}
(with Allen Knutson and Mark Shimozono) \emph{Four positive formulae
	for type~$A$ quiver polynomials}, Inventiones Mathematicae
	\textbf{166} no.~2, 229--325.  \textsf{arXiv:math.AG/0308142}

\bibitem{simpJumps}
(with Laura Matusevich) \emph{Combinatorics of rank jumps in
	simplicial hypergeometric systems}, Proceedings of the
	American Mathematical Society \textbf{134}, 1375--1381.
	\textsf{arXiv:math.AC/0402071}

\bibitem{simplicialPoset}
(with Vic Reiner) \emph{Stanley's simplicial poset conjecture, after
	M.\ Masuda}, Communications in Algebra \textbf{34} (2006),
	no.~3, 1049--1053.
     
\bibitem{rankJumps}\bibyear{2005}
(with Laura Matusevich and Uli Walther) \emph{Homological methods for
	hypergeometric families}, Journal of the American Mathematical
	Society \textbf{18}, no.~4, 919--941.  \textsf{arXiv:math.AG/0406383}

\bibitem{kquiv}
\emph{Alternating formulas for $K$-theoretic quiver polynomials},
	Duke Mathematical Journal \textbf{128}, 1--17.
	\textsf{arXiv:math.CO/0312250}

\bibitem{grobGeom}
(with Allen Knutson) \emph{Gr\"obner geometry of Schubert
	polynomials}, Annals of Mathematics \textbf{161}, 1245--1318.
	\textsf{arXiv:math.AG/0110058}

\bibitem{injAlg}
(with David Helm) \emph{Algorithms for graded injective resolutions
	and local cohomology over semigroup rings}, Journal of
	Symbolic Computation \textbf{39}, 373--395.
	\textsf{arXiv:math.AC/0309256}

\bibitem{deform}
(with Mikhail Kogan) \emph{Toric degeneration of Schubert varieties
	and Gelfand--Tsetlin polytopes}, Advances in Mathematics
	\textbf{193}, no.~1, 1--17.  \textsf{arXiv:math.AG/0303208}

\bibitem{reciprocal}
(with Vic Reiner) \emph{Reciprocal domains and Cohen--Macaulay
	$d$-complexes in ${\mathbb R}^d$}, The Electronic Journal of
	Combinatorics \textbf{11(2)}, \#N1 (9~pages).
	\textsf{arXiv:math.CO/0408169}

\bibitem{subword}\bibyear{2004}
(with Allen Knutson) \emph{Subword complexes in Coxeter groups},
	Advances in Mathematics \textbf{184}, 161--176.
	\textsf{arXiv:math.CO/0309259}

\bibitem{bassNumbers}\bibyear{2003}
(with David Helm) \emph{Bass numbers of semigroup-graded local
	cohomology}, Pacific Journal of Mathematics~\textbf{209},
	no.~1, 41--66.  \textsf{arXiv:math.AG/0010003}

\bibitem{mitosis}
\emph{Mitosis recursion for coefficients of Schubert polynomials},
	Journal of Combinatorial Theory, Series~A~\textbf{103},
	223--235.  \textsf{arXiv:math.CO/0212131}

\bibitem{cmQuotients}\bibyear{2002}
\emph{Cohen--Macaulay quotients of normal semigroup rings via
	irreducible resolutions}, Mathematical Research
	Letters~\textbf{9}, no.~1, 117--128.
	\textsf{arXiv:math.AC/0110096}

\bibitem{planarGraphs}
\emph{Planar graphs as minimal resolutions of trivariate monomial
	ideals}, Documenta Mathematica~\textbf{7}, 43--90.
	(electronically published:
	\textsf{http:/$\!$/www.math.uiuc.edu/documenta/vol-07/03.html})
  
% \bibitem{gmDuality}
% \emph{Graded Greenlees--May duality and the \v Cech hull}, Local
% 	cohomology and its applications (Guanajuato, 1999), 233--253,
% 	Lect. Notes in Pure and Appl. Math., vol. 226, Dekker, New
% 	York.

\bibitem{emThesis}\bibyear{2000}
\emph{Resolutions and duality for monomial ideals}, Ph.D. thesis,
	University of California at Berkeley.

\bibitem{alexDual}
\emph{The Alexander duality functors and local duality with monomial
	support}, Journal of Algebra \textbf{231}, 180--234.

\bibitem{generic}
(with Bernd Sturmfels and Kohji Yanagawa) \emph{Generic and cogeneric
	monomial ideals}, Journal of Symbolic Computation~\textbf{29},
	691--708.

\bibitem{icos}
\emph{Icosahedra constructed from congruent triangles}, Discrete
	and Computational Geometry~\textbf{24}, no.~2--3, 437--451.
  
% \bibitem{monIdealPlanGraph}\bibyear{1999}
% (with Bernd Sturmfels) \emph{Monomial ideals and planar graphs}, in
% 	Applied Algebra, Algebraic Algorithms and Error-Correcting
% 	Codes,
% %	[M.~Fossorier, H.~Imai, S.~Lin and A.~Poli, eds.], Proceedings
% %	of AAECC-13 (Honolulu, Nov.  1999),
% 	\textsl{Springer Lect. Notes in Comp. Sci.} \textbf{1719}
% 	(1999), 19--28.

% \bibitem{oldAlexDual}\bibyear{1998}
% \emph{Alexander duality for monomial ideals and their resolutions}.
% 	\textsf{arXiv:math.AC/9812095}

\bibitem{multiplicities}\bibyear{1998}
\emph{Multiplicities of ideals in noetherian rings}.  Beitr\"age zur
	Algebra und Geometrie~\textbf{39}(1) 47--51.

%%%%%%%%%%%%%%%%%%%%%%%%%%%%%%%%%%%%%%%%%%%%%%%%%%%%%%%%%%%%%%%%%%%%%%
\bibsubsec{\sf Conference publications (peer-reviewed and/or invited)}
%%%%%%%%%%%%%%%%%%%%%%%%%%%%%%%%%%%%%%%%%%%%%%%%%%%%%%%%%%%%%%%%%%%%%%

\bibitem{hornMEGA}\bibyear{2007}
(with Alicia Dickenstein and Laura Matusevich) \emph{Extended
	abstract: Binomial $D$-modules}, Proceedings MEGA (Effective
	Methods in Algebraic Geometry), Strobl, Austria, 2007,
	13~pages.
%	24th - 30th June 2007
%	http://www.ricam.oeaw.ac.at/mega2007/

\bibitem{multIdealSumsMFO}
(with Shin-Yao Jow) \emph{Extended abstract: Cellular resolutions of
	multiplier ideals of sums}, in \emph{Top\-ological and geometric
	combinatorics}, abstracts from the workshop held
	January~28--February~3, 2007, organized by Anders Bj\"orner,
	Gil Kalai, and G\"unter Ziegler, Oberwolfach reports, 3~pages.

\bibitem{rankJumpsMFO}\bibyear{2006}
(with Laura Matusevich and Uli Walther) \emph{Extended abstract:
	Homological methods for hypergeometric families}, in
	\emph{Convex and algebraic geometry}, abstracts from the
	workshop held January~29--February~4, 2006, organized by
	Klaus Altmann, Victor Batyrev, and Bernard Teissier,
	Oberwolfach reports, 3 pages.

\bibitem{injAlgMEGA}\bibyear{2003}
(with David Helm) \emph{Extended abstract: Algorithms for graded
	injective resolutions and local cohomology over semigroup
	rings}, Proceedings MEGA (Effective Methods in Algebraic
	Geometry), Kaiserslautern, Germany, 2003, 5~pages.
%	10th - 14th June 2003
%	http://www.mathematik.uni-kl.de/~wwwagag/workshops/mega-03/

\bibitem{gm}\bibyear{2002}
\emph{Graded Greenlees--May duality and the \v Cech hull}, Local
	cohomology and its applications (Guanajuato, 1999), Lecture
	Notes in Pure and Appl. Math., vol. 226, Dekker, New York,
	233--253.

\bibitem{grobGeomFPSAC}
(with Allen Knutson) \emph{Extended abstract: Gr\"obner geometry of
	Schubert polynomials}, Proceedings FPSAC (Formal Power Series
	and Algebraic Combinatorics), Melbourne 2002, 10 pages.

\bibitem{AAECC}\bibyear{1999}
(with Bernd Sturmfels) \emph{Monomial ideals and planar graphs}, in
	Applied Algebra, Algebraic Algorithms and Error-Correcting
	Codes, [M.~Fossorier, H.~Imai, S.~Lin and A.~Poli, eds.],
	Proceedings of AAECC-13 (Honolulu, Nov.  1999), {\sl Springer
	Lecture Notes in Computer Science} \textbf{1719}, 19--28.

\end{thebibliography}
%\bibliographystyle{amsalpha}
%\bibliography{biblio}
\renewcommand\refname{}
\begin{thebibliography}{Mat01b}


%%%%%%%%%%%%%%%%%%%%%%%%%%%%%%%%%%%%%%%%%%%%%%%%%%%%%%%%%%%%%%%%%%%%%%
\bibsec{\textbf{Cited publications by other authors}}%%%%%%%%%%%%%%%%%

\bibitem[Ado94]{Ado94}
Alan Adolphson, \emph{Hypergeometric functions and rings generated by
  monomials}, Duke Math. J. \textbf{73} (1994), no.~2, 269--290.

\bibitem[Ale48]{Ale48}
A.~D. Alexandrov, \emph{Vnutrennyaya geometriya vypuklykh
  poverkhnostey {\rm (in Russian)}}, {\mbox{M.--L.}: Gostekhizdat},
  1948; English translation: \emph{Selected works. Intrinsic geometry
  of convex surfaces}, Vol.~2, Chapman \& Hall/CRC, Boca Raton, FL, 2005.

\bibitem[Ale50]{Ale50}
A. D. Alexandrov, \emph{Convex polyhedra}, Springer Monographs in
  Mathematics.  Springer--Verlag, Berlin, 2005.  Translated from the
  1950 Russian edition by N.~S.~Dairbekov, S.~S.~Kutateladze and
  A.~B.~Sossinsky, with comments and bibliography by V.~A.~Zalgaller
  and appendices by L.~A.~Shor and Yu.~A.~Volkov.

\bibitem[AO92]{AO92}
Boris Aronov and Joseph O'Rourke, \emph{Nonoverlap of the star
  unfolding}, Discrete Comput. Geom. \textbf{8} (1992), no.~3,
  219--250.

\bibitem[BPS98]{BPS98}
Dave Bayer, Irena Peeva, and Bernd Sturmfels, \emph{Monomial
  resolutions}, Math. Res. Lett. \textbf{5} (1998), no.~1-2, 31--46.

\bibitem[BI06]{BI06}
Alexander I. Bobenko and Ivan Izmestiev, \emph{Alexandrov's theorem,
  weighted Delaunay triangulations, and mixed volumes}, preprint.
  \textsf{arXiv:math.DG/0609447}

\bibitem[Bri02]{brion02}
M. Brion, \emph{Positivity in the Grothendieck group of complex flag
  varieties}, Journal of Algebra \textbf{258} (2002), no. 1,
  137--159.

\bibitem[Buc02]{buchLR}
A.~S.~Buch, \emph{A Littlewood-Richardson rule for the \K-theory ring
  of Grassmannians,} Acta Math., {\bf 189} (2002), no.~1, 137--159.
  \texttt{math.AG/0004137}

\bibitem[Buc02a]{buch02}
Anders~S. Buch, \emph{Grothendieck classes of quiver varieties},
  Duke Math. J. \textbf{115} (2002), no.~1, 75--103.

\bibitem[Buc03]{buchAlt}
Anders~S. Buch, \emph{Alternating signs of quiver coefficients},
  J. Amer. Math. Soc. \textbf{18} (2005), no. 1, 217--237 (electronic). 

\bibitem[BF99]{BF99}
Anders~Skovsted Buch and William Fulton, \emph{Chern class formulas
  for quiver varieties}, In- vent. Math. \textbf{135} (1999), no.~3,
  665--687.

\bibitem[CdG06]{CdG06}
Richard Casey, Hidde de Jong, and Jean-Luc Gouz\'e,
\emph{Piecewise-linear models of genetic regulatory networks:
  equilibria and their stability}, J. Math. Biol. \textbf{52} (2006),
  no.~1, 27--56.

\bibitem[CL03]{CL03}
B. Chow and F. Luo, \emph{Combinatorial Ricci flows on surfaces},
  J. Differential Geom. \textbf{63} (2003), 97--129.

\bibitem[ER98]{ER98}
John~A. Eagon and Victor Reiner, \emph{Resolutions of
  {S}tanley--Reisner rings and {A}lexander duality}, J. Pure
  Appl. Algebra \textbf{130} (1998), no.~3, 265--275.

\bibitem[ES96]{binomialIdeals}
David Eisenbud and Bernd Sturmfels, \emph{Binomial ideals}, Duke
   Math. J. \textbf{84} (1996), no.~1, 1--45.

\bibitem[Erd50]{erdelyi}
Arthur Erd\'elyi, \emph{Hypergeometric functions of two variables},
  Acta Math. \textbf{83} (1950), 131--164.

\bibitem[Fel01]{felsner01}
Stefan Felsner, \emph{Convex drawings of planar graphs and the order
  dimension of 3-polytopes}, Order \textbf{18} (2001), no.~1, 19--37.

\bibitem[Fel03]{felsner03}
Stefan Felsner, \emph{Geodesic embeddings and planar graphs}, Order
  \textbf{20} (2003), no.~2, 135--150.

\bibitem[GGZ87]{GGZ87}
I.~M. Gel{$'$}fand, M.~I. Graev, and A.~V. Zelevinski\u{\i},
  \emph{Holonomic systems of equations and series of hypergeometric
  type}, Dokl. Akad. Nauk SSSR \textbf{295} (1987), no.~1, 14--19.

\bibitem[GKZ89]{GKZ89}
I.~M. Gel{$'$}fand, A.~V. Zelevinski\u{\i}, and M.~M. Kapranov,
  \emph{Hypergeometric functions and toric varieties},
  Funktional. Anal. i Prilozhen. \textbf{23} (1989), no.~2, 12--26.
  Correction in ibid, \textbf{27} (1993), no.~4,~91.

\bibitem[Gli05]{glick05}
David Glickenstein, \emph{A combinatorial Yamabe flow in three
  dimensions}, Topology \textbf{44} (2005), no.~4, 791--808.

\bibitem[GL96]{GL96}
N.~Gonciulea and V.~Lakshmibai, \emph{Degenerations of flag and
  {S}chubert varieties to toric varieties}, Transform. Groups \textbf{1}
  (1996), no.~3, 215--248.

\bibitem[Gra01]{grahamPos}
William Graham, \emph{Positivity in equivariant Schubert calculus},
  Duke Math. J. \textbf{109} (2001), no.~3, 599--614.

\bibitem[GM92]{gm92}
John P.~C. Greenlees and J.~Peter May, \emph{Derived functors of
  \mbox{${I}$-adic} completion and local homology}, J. Algebra \textbf{149}
  (1992), no.~2, \mbox{438--453}.
% \MR{93h:13009}

\bibitem[Har70]{HarCofinite}
Robin Hartshorne, \emph{Affine duality and cofiniteness},
  Invent. Math. \textbf{9} (1969/1970), 145--164.

\bibitem[Hor1889]{horn89}
J.~Horn, \emph{ {\"Uber die konvergenz der hypergeometrischen {R}eihen
  zweier und dreier Ver\"ander\-lichen}}, Math. Ann. \textbf{34} (1889),
  544--600.

\bibitem[LS82]{LSpolySchub}
Alain Lascoux and Marcel-Paul Sch{\"u}tzenberger, \emph{Polyn\^omes de
  {S}chubert}, C. R. Acad. Sci. Paris S\'er. I Math. \textbf{294}
  (1982), no.~13, 447--450.

\bibitem[Man79]{manning79}
Anthony Manning, \emph{Topological entropy for geodesic flows},
  Ann. of Math. (2) \textbf{110} (1979), no.~3, 567--573.

\bibitem[Mas03]{masuda03}
Mikiya Masuda, \emph{$h$-vectors of Gorenstein$\!$* simplicial
  posets}, 2003.  \textsf{arXiv:math.CO/0305203}

\bibitem[MMP87]{MMP87}
J.~S.~B. Mitchell, D.~M. Mount, and C.~H. Papadimitriou,  The
  discrete geodesic problem, \emph{SIAM J.  Comp.} \textbf{16}
  (1987), no.~4, 647--668.

\bibitem[PS98]{PS98}
Irena Peeva and Bernd Sturmfels, \emph{Generic lattice ideals},
  J. Amer. Math. Soc. \textbf{11} (1998), no.~2, 363--373.

\bibitem[SS86]{SS}
M. Sharir and A. Schorr, On shortest paths in polyhedral spaces,
  \emph{SIAM J. Comp.} \textbf{15} (1986), no.~1, 193--215.

\bibitem[Sta74]{Sta74}
R.~P. Stanley, \emph{Combinatorial reciprocity theorems}, Advances in
  Math. \textbf{14} (1974), 194--253.

\bibitem[Sta91]{Sta91}
Richard~P. Stanley, \emph{{$f$}-vectors and {$h$}-vectors of
  simplicial posets}, J. Pure Appl. Algebra \textbf{71} (1991),
  no.~2-3, 319--331.

\bibitem[Vol68]{Vol68}
Y. A. Volkov, \emph{An estimate for the deformation of a convex
  surface in dependence on the variation of its intrinsic
  metric}, Ukrain. Geometr. Sb. \textbf{5-6} (1968), 44--69.

\bibitem[VP71]{VP71}
Ju.~A. Volkov and E.~G. Podgornova, The cut locus of a polyhedral
  surface of positive curvature (in Russian), \emph{Ukrainian
  Geometric Sbornik} \textbf{11} (1971), 15--25.

\end{thebibliography}
%%%%%%%%%%%%%%%%%%%%%%%%%%%%%%%%%%%%%%%%%%%%%%%%%%%%%%%%%%%%%%%%%%%%%%
\end{document}%%%%%%%%%%%%%%%%%%%%%%%%%%%%%%%%%%%%%%%%%%%%%%%%%%%%%%%%
%%%%%%%%%%%%%%%%%%%%%%%%%%%%%%%%%%%%%%%%%%%%%%%%%%%%%%%%%%%%%%%%%%%%%%
