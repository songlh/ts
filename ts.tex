\documentclass[10pt]{article}
\usepackage{geometry}
\usepackage{amsmath}

\usepackage{url}
\addtolength{\oddsidemargin}{-.4in}
\addtolength{\evensidemargin}{-.4in}
\addtolength{\textwidth}{0.8in}

\addtolength{\topmargin}{-0.3in}
\addtolength{\textheight}{1.3in}
\usepackage[numbers, sort]{natbib}

\usepackage{paralist}
\newenvironment{ParaEnum}[0]{\begin{inparaenum}[(1)]}{\end{inparaenum}}

\title{\vspace{-.7in}\bf{Linhai Song - Teaching Statement\vspace{-.4in}}}
%\author{Guoliang Jin}
\date{}
\begin{document}
\maketitle\vspace{-.2in}

I understand the importance of teaching, 
which can process power to influence and eventually shape the whole society. 
I personally benefit from numerous great people, who taught me and mentored me. 
I think the best way to pay back is to pay what I have learned forward to more junior people. 
As a faculty member, I would gain the chance to help and watch my students to learn, to grow, and to mature. 
It is exciting to imagine the next great mind prepared in my course.
The opportunity to teach is one of key factors making me to choose faculty career. 

My teaching interests grow from my tutoring experience. 
To improve my spoken English, 
I served as free tutor in GUTS (Greater University Tutor Service) of University of Wisconsin-Madison in the first semester of my Ph.D. study. 
I taught Paul Dolan Introduction to Statistical Methods (Stat 301) and Nathalie Cheng Computer Graphics (CS 559). 
Stat 301 is to provide a basic working knowledge of the concepts and techniques used in statistics. 
I helped Paul review what he learned and answered his specific questions through our weekly meeting. 
I learned that a right tutoring method could greatly help explanation. 
As an example, I found that inferring maths formula together with Paul can help him understand difficult statistical concepts and methods clearly. 
CS 559 is designed for senior undergraduate students and graduate students. 
I primarily helped Nathalie finish her programming assignment: 
I helped understand the assignment requirement, 
configured programming environment, 
tested graphics libraries, 
and offered high-level design hints and low-level implementation tips. 
I learned that quickly getting visible results can 
greatly motivate students to finish and improve programming assignments. 



\vspace{0.15in}
\paragraph*{Courses.} As a faculty member, I am interested in both entry-level and advanced courses. 
My research experience and technical background allow me to teach courses in programming languages, operating systems, and software engineering.
Such a course will not only teach classics under each topic, but also provide hands-on experience in cutting-edging techniques.
One important thing I learn from my weekly meeting with my advisor is to ask questions periodically to check students' understanding.
I will also do this in my classes and especially encourage those shy students to answer to keep everyone engaged. 
Besides this, I will encourage students to ask me questions. 
I will figure out insights behind these questions, and hint on point to guide my teaching. 
There are many high-quality open-source projects available, 
driving core business behind high-technique corporations. 
I plan to design course projects based on these open-source projects 
to expose students to the real world and help them gain working knowledge of state-of-the-arts. 

\vspace{0.15in}
\paragraph*{Mentoring.}One exciting aspect of academia jobs is to interact with junior students, 
educate them and learn new ideas and thinking from them. 
After becoming a senior Ph.D. student, I began to help mentor junior students in our group. 
My responsibility includes helping define the goal of their research projects, 
helping breaking projects into small pieces of milestones, 
and solving detailed technical problems. 
Many of mentored projects lead to paper submissions and publications in top conference. 
I learned a lot from this mentoring process: 

%\newpage
%\bibliographystyle{plainyr-rev}
%\bibliography{rs}
\end{document}
