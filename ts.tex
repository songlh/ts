\documentclass[10pt]{article}
\usepackage{geometry}
\usepackage{amsmath}

\usepackage{url}
\addtolength{\oddsidemargin}{-.4in}
\addtolength{\evensidemargin}{-.4in}
\addtolength{\textwidth}{0.8in}

\addtolength{\topmargin}{-0.3in}
\addtolength{\textheight}{1.3in}
\usepackage[numbers, sort]{natbib}

\usepackage{paralist}
\newenvironment{ParaEnum}[0]{\begin{inparaenum}[(1)]}{\end{inparaenum}}

\title{\vspace{-.7in}\bf{Linhai Song - Teaching Statement\vspace{-.4in}}}
%\author{Guoliang Jin}
\date{}
\begin{document}
\maketitle\vspace{-.2in}

I understand the importance of teaching, which has the power to influence and eventually shape our whole society. 
I have personally benefited from numerous great people who have taught me and mentored me. 
I believe the best way to pay this back is to pay what I have learned forward to junior people. 
As a faculty member, I would have the chance to help my students to learn, grow, and mature. 
It is exciting to imagine the next great mind prepared in my course. 
The opportunity to continue teaching is one of the key factors that has led me to choose a career as a faculty member.

\vspace{-0.1in}
\paragraph*{Experiences.} 
My teaching interests stem from my tutoring experience. 
To improve my spoken English, I served as a
free tutor in the Greater University Tutor Service (GUTS) of the University of Wisconsin-Madison during the my first semester of Ph.D. study. 
I taught Paul Dolan Introduction to Statistical Methods (Stat 301) and Nathalie Cheng
Computer Graphics (CS 559). 
Stat 301 aims to provide a basic working knowledge of the concepts and techniques
used in statistics. 
I held weekly meetings with Paul to help review what he learned and answer his specific questions, and I learned that the right tutoring method can greatly help explanations. 
As an example, I found that inferring math formulas together with Paul helped him understand difficult statistical concepts and methods more clearly. 
CS 559 was a different type of course, designed for senior undergraduate students and graduate students. 
I primarily helped Nathalie finish her programming assignment: I helped her understand the assignment requirements, 
showed her how to configure the programming environment, 
explained graphics libraries to her, 
and offered her high-level program design hints and low-level implementation tips. 
I learned that quickly getting visible results can greatly motivate students to finish and improve their programming assignments. 
Although these two experiences are quite
different, both of them were enjoyable and rewarding, and they are the major force that has driven me to pursue opportunities to continue teaching.


\vspace{-0.1in}
\paragraph*{Teaching.}
As a faculty member, I am interested in teaching both entry-level and advanced courses. My research experience and technical background will allow me to teach courses in programming languages, operating systems, and software engineering. Such courses will not only teach the fundamental concepts under each topic, but also provide hands-on experience in cutting-edge techniques.

I have gradually built my teaching philosophy through many years of learning, researching, and interacting with other people. 
One important thing I have learned from my weekly meetings with my advisor is to periodically ask questions; I will also do this in my classes to check students' understanding. 
I will especially encourage shy students to answer in order to keep everyone engaged in my classes. 
Encouraging students to ask questions is also a great way to collect feedback, and I will analyze students' questions to figure out the insights behind these questions and
hints on point to guide my teaching. 
As for course assignments, there are many high-quality open-source projects
available that drive the core businesses behind high-tech corporations, and I plan to design course projects 
based on these projects to expose students to the real world and help them gain a working knowledge of state-of-the-art technologies.


\vspace{-0.1in}
\paragraph*{Mentoring.}
One exciting aspect of academic jobs is the chance to interact with junior students, educate them, and learn new ideas and ways of thinking from them. 
As a senior Ph.D. student, I began to work with junior students in
our group and helped mentor them. 
My responsibilities include leading brainstorming discussions to explore all
possibilities of their research projects, helping to break their projects into small pieces and milestones, 
teaching necessary background information, and helping to solve technical problems. 
Many of these mentored projects have
led to paper submissions and publications in top conferences.

I have learned a lot from this mentoring process. First, students may have different expectations of their graduate schools, from preparing them for job-hunting to exploring research interests to simply pure curiosity. 
It is important to understand students' expectations early, encourage them to think about plans after graduate school, help them explore different options to set up concrete career goals, and encourage them to work towards these goals.
Second, different students have different personalities and different levels of maturity. Communication with students must be flexible. It is also important to encourage students to jump out of their comfort zones to improve communication skills. 
Third, a long-term project can easily exhaust students. It is important to break a large project
down into small pieces of testable milestones so that students can stay motivated with the feeling of setting something down. 
This method can also catch students' mistakes in early stages and allow them to learn from their mistakes. 
Fourth, teaching knowledge is important, but what is more important is teaching students the whole process of problem-solving and helping students grow into independent engineers and researchers.







%\newpage
%\bibliographystyle{plainyr-rev}
%\bibliography{rs}
\end{document}
