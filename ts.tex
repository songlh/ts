\documentclass[10pt]{article}
\usepackage{geometry}
\usepackage{amsmath}

\usepackage{url}
\addtolength{\oddsidemargin}{-.4in}
\addtolength{\evensidemargin}{-.4in}
\addtolength{\textwidth}{0.8in}

\addtolength{\topmargin}{-0.3in}
\addtolength{\textheight}{1.3in}
\usepackage[numbers, sort]{natbib}

\usepackage{paralist}
\newenvironment{ParaEnum}[0]{\begin{inparaenum}[(1)]}{\end{inparaenum}}

\title{\vspace{-.7in}\bf{Linhai Song - Teaching Statement\vspace{-.4in}}}
%\author{Guoliang Jin}
\date{}
\begin{document}
\maketitle\vspace{-.2in}

I understand the importance of teaching, 
which can process power to influence and eventually shape the whole society. 
I personally benefit from numerous great people, who taught me and mentored me. 
I think the best way to pay back is to pay what I have learned forward to more junior people. 
As a faculty member, I would gain the chance to help and watch my students to learn, to grow, and to mature. 
It is exciting to imagine the next great mind prepared in my course.
The opportunity to continue teaching is one of key factors making me choose faculty career. 

\vspace{-0.1in}
\paragraph*{Experiences.} My teaching interests grow from my tutoring experience. 
To improve my spoken English, 
I served as free tutor in GUTS (Greater University Tutor Service) of University of Wisconsin-Madison in the first semester of my Ph.D. study. 
I taught Paul Dolan Introduction to Statistical Methods (Stat 301) and Nathalie Cheng Computer Graphics (CS 559). 
Stat 301 aims to provide a basic working knowledge of the concepts and techniques used in statistics. 
I held weekly meeting with Paul to help review what he learned and answered his specific questions. 
I learned that a right tutoring method could greatly help explanation. 
As an example, I found that inferring maths formula together with Paul can help him understand difficult statistical concepts and methods clearly. 
CS 559 is designed for senior undergraduate students and graduate students. 
I primarily helped Nathalie finish her programming assignment: 
I helped understand the assignment requirement, 
showed her how to configure programming environment, 
explained graphics libraries to her, 
and offered her high-level program design hints and low-level implementation tips. 
I learned that quickly getting visible results can 
greatly motivate students to finish and improve their programming assignments. 
Although these two experiences are quite different, both of them are enjoyable and rewarding, 
and they are the major force to drive me to pursue opportunities to continue teaching.


\vspace{-0.1in}
\paragraph*{Teaching.}
As a faculty member, I am interested in teaching both entry-level and advanced courses. 
My research experience and technical background allow me to teach courses in \textit{programming languages}, \textit{operating systems}, and \textit{software engineering}.
Such a course will not only teach fundamental concepts under each topic, but also provide hands-on experience in cutting-edging techniques.

I gradually built my teaching philosophy through so many years' learning, researching, and interaction with other people. 
One important thing I learn from my weekly meeting with my advisor is to periodically ask questions. 
I will also do this in my classes to check students' understanding. 
I will especially encourage those shy students to answer to keep everyone engaged in my classes. 
Encouraging students to ask questions is a great way to collect feedbacks. 
I will analyze students' questions to figure out insights behind these questions, 
and hint on point to guide my teaching. 
There are many high-quality open-source project available, 
which drive core business behind high-technique corporations. 
I plan to design course projects based on these projects to expose students to the real world 
and help them gain working knowledge of state-of-the-arts. 

\vspace{-0.1in}
\paragraph*{Mentoring.}One exciting aspect of academia jobs is to interact with junior students, 
educate them and learn new ideas and thinking from them. 
After being a senior Ph.D. student, I began to work with junior students in our group and helped mentor them. 
My responsibility includes leading brainstorming discussions to explore all possibilities of their research projects, 
helping breaking their projects into small pieces of milestones, 
teaching necessary background, and helping solving technical problems.
Many of mentored projects lead to paper submissions and publications in top conference. 

I learned a lot from this mentoring process. 
First, students may have different expectations of their graduate schools. 
It could be to prepare for their job hunting, to explore research interests, or simply pure curiosity. 
It is important to understand students' expectations early, encourage them to think about plans after graduate schools, 
help them explore different options to set up concrete career goals, 
and encourage them to work towards the goals. 
Second, different students have different personality and different levels of maturity.
Communicating with students must be flexible. 
It is also important to encourage students jump out of their comfortable zones to improve communication skills. 
Third, a long-term project can easily exhaust students. 
It is important to break a large project into small pieces of testable milestones. 
Students can keep motivated with the feeling of settling something down.
It can also capture students' mistakes in early stage, and allow them to learn from their mistakes.  
Four, teaching knowledges is important, but what is more important is to teach students 
the whole process of problem-solving and 
help students grow into an independent engineer and researcher. 



%\newpage
%\bibliographystyle{plainyr-rev}
%\bibliography{rs}
\end{document}
